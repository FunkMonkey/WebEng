% Document Prämbels
%\documentclass[12pt,a4paper,titlepage,abstracton]{scrreprt}

\documentclass[%
	pdftex,%              PDFTex verwenden
	a4paper,%             A4 Papier
	oneside,%             Einseitig
	%bibtotocnumbered,%    Literaturverzeichnis nummeriert einfügen
	%idxtotoc,%            Index ins Verzeichnis einfügen
	%halfparskip,%        Europäischer Satz mit abstand zwischen Absätzen
	parskip=half,
	chapterprefix,%       Kapitel anschreiben als Kapitel
	headsepline,%         Linie nach Kopfzeile
	footsepline,%         Linie vor Fusszeile
	11pt%                 Grössere Schrift, besser lesbar am bildschrim
]{memoir} % maybe use memoir

\setlength{\parindent}{0pt}
\setlength{\parskip}{1ex plus 0.5ex minus 0.2ex}

\usepackage[utf8]{inputenc}

%\usepackage{microtype}

%\usepackage{graphicx}
\usepackage[round, numbers]{natbib}
% Silbentrennung
\usepackage[ngerman]{babel}
% Umlaute
%\usepackage[ansinew]{inputenc}
% Stickwortverzeichnis
\usepackage{makeidx}

\usepackage{wrapfig}

\usepackage{multirow}

\makeindex

%\usepackage{xcolor}
\usepackage{color,listings} %bindet das Paket Listings ein


% Der lstset-Befehl ermöglicht haufenweise Einstellungen zur Formatierung
\definecolor{mygray}{gray}{0.82}

\lstset{numbers=none, % keine Zeilennummern
tabsize=3, % Tabulatorgrösse: 3 Zeichen
breaklines=true, % zu lange Zeilen werden umbrochen
aboveskip=1em, % Abstand nach oben
belowskip=0.3em, % Abstand nach unten
basicstyle=\tiny\ttfamily, % Schriftgrösse small, Typewriter-Font
framerule=1pt, % keinen Rand
backgroundcolor=\color{mygray}, % helles grau als Hintergrund
framexrightmargin=0.7em, % Hintergrund ragt leicht in den Seitenrand
framexleftmargin=0.7em, % Hintergrund ragt leicht in den Seitenrand
captionpos=b % Beschriftung ist unterhalb
columns=fullflexible % damit Quellcode einfach rauskopiert werden kann
}

%Caption
%\usepackage[options]{caption}
\usepackage[small]{caption}

% Code Auslistungen
\renewcommand{\lstlistlistingname}{Quellcodeverzeichnis}
\renewcommand{\lstlistingname}{Quellcode}


%\newcommand{\hdepagenumber}{\thispagestyle{empty}}
%\newcommand{\@showpagenumber}{\thispagestyle{myheadings}}
%\usepackage{glossary}
%\makeglossary
\usepackage[german]{nomencl}
\makenomenclature
\renewcommand{\nomlabel}[1]{\textbf{#1}}
\renewcommand{\nomname}{Glossar}


\usepackage{graphicx}
\makeatletter
\def\ScaleIfNeeded{%
  \ifdim\Gin@nat@width>\linewidth
  \linewidth
  \else
  \Gin@nat@width
  \fi
}
\makeatother

\usepackage{floatflt}

\usepackage{todonotes}
\usepackage{geometry}
\usepackage{ulem}
\geometry{a4paper, bottom=3cm} 

%++++++++++++++++++++++++++++++++++++++++++++++++++++++++++++++++++
%+++ Zeilenabstand ++++++++++++++++++++++++++++++++++++++++++++++++
%++++++++++++++++++++++++++++++++++++++++++++++++++++++++++++++++++
%\usepackage{setspace}
% eineinhalbfacher Zeilenabstand. Dies ist nicht gleich wie Zeilenabstand
% 1.5 in üblicher Textverarbeitungssoftware!
\OnehalfSpacing

\usepackage{microtype}

%\usepackage{minted}


%++++++++++++++++++++++++++++++++++++++++++++++++++++++++++++++++++
%+++ Verlinkung +++++++++++++++++++++++++++++++++++++++++++++++++++
%++++++++++++++++++++++++++++++++++++++++++++++++++++++++++++++++++
% Doku: http://www.ctan.org/tex-archive/macros/latex/contrib/hyperref/hyperref.pdf
\usepackage[%
colorlinks, % verwende farbige Links
linkcolor=black, % Linkfarbe ist blau
bookmarks, % erstelle Bookmarks der Links
bookmarksopen, % Bookmarks werden beim Öffnen des Dokumentes ebenfalls geöffnet
urlcolor=black, % Hyperlinks sind blau 
bookmarksnumbered, % Bookmarks sind nummeriert
final % Endversion 
]{hyperref}

\bibpunct{}{}{,}{n}{}{;}

\usepackage{hyperref}

% commands
\newcommand{\textprog}[1]{\textit{#1}}

\newcommand{\myLink}[1]
{
	\textbf{#1}
}

%\newcommand{\myTOCLink}[1]{\textbf{Kapitel \ref{#1}: \nameref{#1}}}

\newcommand{\myRefChapter}[1]{\textbf{\hyperref[#1]{Kapitel ~\ref*{#1}: ~\nameref{#1}}}}
\newcommand{\myRefSection}[1]{\textbf{\hyperref[#1]{Abschnitt ~\ref*{#1}: ~\nameref{#1}}}}

\newcommand{\myNameref}[1]{\textbf{\nameref{#1}}}

\newcommand{\myNomRef}[2]{\textbf{\textsl{\hyperref[#2]{#1}}}}

\newcommand{\myBookCiteTook}[1]{\textsuperscript{[entnommen aus \citeauthor{#1} \citeyear{#1}]}}

\newcommand{\myBookCite}[2]{\footnote{[\cite{#1}] \citeauthor*{#1} \citeyear{#1}, #2}}
\newcommand{\myBookCiteSense}[2]{\footnote{sinngemäß nach [\cite{#1}] \citeauthor*{#1} \citeyear{#1}, #2}}

\newcommand{\myBookCiteNoP}[1]{\footnote{[\cite{#1}] \citeauthor*{#1} \citeyear{#1}}}
\newcommand{\myBookCiteSenseNoP}[1]{\footnote{sinngemäß nach [\cite{#1}] \citeauthor*{#1} \citeyear{#1}}}

%\newcommand{\myWebCiteNoP}[1]{\footnote{[\cite{#1}] \citetalias{#1}}}
%\newcommand{\myWebCiteSenseNoP}[1]{\footnote{sinngemäß nach [\cite{#1}] \citetalias{#1}}}

%\newcommand{\myWebCiteSense}[2]{\footnote{sinngemäß nach [\cite{#1}] \citetalias{#1}, #2}}
%\newcommand{\myWebCite}[2]{\footnote{[\cite{#1}] \citetalias{#1}, #2}}


%\defcitealias{xblspec}{XML Binding Language (XBL) 2.0}
%\defcitealias{cssspec}{Cascading Style Sheets Level 2 Revision 1 (CSS 2.1) Specification}
%\defcitealias{ecmaspec}{ECMAScript Language Specification}
%\defcitealias{opensearch}{Specifications \texttt{>} OpenSearch \texttt{>} 1.1 \texttt{>} Draft 4}
%\defcitealias{filespec}{Uniform Resource Locators (URL)}
%\defcitealias{urispec}{Uniform Resource Identifier (URI): Generic Syntax}


\hypersetup{
    colorlinks,%
    citecolor=black,%
    filecolor=black,%
    linkcolor=black,%
    urlcolor=black
}

% JS

\definecolor{lightgray}{rgb}{.9,.9,.9}
\definecolor{darkgray}{rgb}{.4,.4,.4}
\definecolor{purple}{rgb}{0.65, 0.12, 0.82}

\lstdefinelanguage{JavaScript}{
  keywords={typeof, new, true, false, catch, function, return, null, catch, switch, var, if, in, while, do, else, case, break},
  keywordstyle=\color{blue}\bfseries,
  ndkeywords={class, export, boolean, throw, implements, import, this},
  ndkeywordstyle=\color{darkgray}\bfseries,
  identifierstyle=\color{black},
  sensitive=false,
  comment=[l]{//},
  morecomment=[s]{/*}{*/},
  commentstyle=\color{purple}\ttfamily,
  stringstyle=\color{red}\ttfamily,
  morestring=[b]',
  morestring=[b]"
}

\lstset{
   language=JavaScript,
   backgroundcolor=\color{lightgray},
   extendedchars=true,
   basicstyle=\footnotesize\ttfamily,
   showstringspaces=false,
   showspaces=false,
   numbers=left,
   numberstyle=\footnotesize,
   numbersep=9pt,
   tabsize=2,
   breaklines=true,
   showtabs=false,
   captionpos=b
}

\lstset{
  literate={ö}{{\"o}}1
           {ä}{{\"a}}1
           {ü}{{\"u}}1
           {ß}{{\ss}}1
}
