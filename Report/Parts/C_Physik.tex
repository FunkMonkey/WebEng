\chapter{Physik}

Für die Physik nutzt die Engine innerhalb des \textprog{PhysicsCore}s die box2dweb-Bibliothek, ein Javascript-Port der C++-Bibliothek Box2D. Der \textprog{PhysicsCore} erzeugt eine eigene Physik-Welt, innerhalb welcher die Physik-Objekte leben.

\section{Plugin\_PhysicsBox}

Der Großteil der Spielobjekte besitzen eine einfache rechteckige Form - die Böden und Hindernisse der Spielwelt, Trigger und Ähnliches. Um für solche Objekte das Hinzufügen von Physik zu erleichtern, habe ich ein einfaches Plugin, das \textprog{Plugin\_PhysicsBox} geschrieben, welches lediglich dem Spiel-Objekt hinzugefügt werden muss um diesem physikalische Eigenschaften zu geben. Das Plugin holt sich Position (\textprog{pos}) und Größe (\textprog{size}) vom zugewiesenen Spielobjekt (basierend auf den Eigenschaften des \textprog{Plugin\_WorldObject3D} und kümmert sich um die Synchronisation von physikalischer Position und Position des Spielobjekts.

Bei Erstellung des Plugins können physikalische Eigenschaften wie Reibung festgelegt werden, ebenso ob es ein statisches oder dynamisches Objekt ist und ob es ein Sensor (registriert Überschneidungen, aber kollidiert nicht) ist.

\section{Kollisionen}

Innerhalb von box2dweb gibt es die Möglichkeit einen eigenen \textprog{ContactListener} zu schreiben und der Physik-Welt zuzuordnen um physikalische Kollisionen auszuwerten. Der \textprog{ContactListener} muss Funktionen für die Auswertung der 4 Phasen einer Kollision implementieren:

beginContact
endContact
preSolve
postSolve
\todo{Tabelle}

Egal welche Objekte miteinander kollidieren, es werden jeweils genau diese 4 Funktionen aufgerufen. Damit die Spiellogik über Kollisionen informiert wird, sind die Funktionen so implementiert, dass sie für die beiden beteiligten Physik-Objekte die entsprechende Funktion (z. B. \textprog{onBeginContact}) aufrufen, sofern diese vom Spiellogik-Programmierer bereit gestellt wurde. Innerhalb dieses Aufrufs kann dann die Kollision ausgewertet werden. Möglich ist dies nur dadurch, dass man innerhalb von Javascript jedem Objekt beliebige Eigenschaften und Funktionen hinzufügen kann.

\begin{lstlisting}[language=JavaScript]
BeginContact: function BeginContact(contact)
{
	var fixA = contact.GetFixtureA();
	var fixB = contact.GetFixtureB();
	
	if(fixA.onBeginContact)
		fixA.onBeginContact(fixA, fixB, contact);
		
	if(fixB.onBeginContact)
		fixB.onBeginContact(fixB, fixA, contact);
},
\end{lstlisting}
\todo{Caption}
