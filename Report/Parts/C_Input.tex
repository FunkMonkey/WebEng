\chapter{Input}

Der Input eines WebGL-basierten Spiels geschieht mit Hilfe der HTML Keyboard- und Mouse-Events, welche mit Event-Listenern abgefangen werden müssen. Dies geschieht innerhalb des \textprog{InputCore}s. Der \textprog{InputCore} sammelt die Events und wertet sie pro Frame aus, sodass der Programmierer an jeder beliebigen Stelle des Programmcodes auf den aktuellen Zustand der Input-Devices zugreifen kann, anstatt wie in HTML üblich nur per Event über eine Veränderung des Inputs informiert zu werden.

Der \textprog{InputCore} tracked die Mouse-Position sowie die Veränderung der Mouse seit dem letzten Frame. Für Keyboard- und Mouse-Tasten werden 4 Zustände gespeichert: up (Taste ist nicht gedrückt), down (Taste ist gedrückt), pressed (Taste wechselte Zustand von nicht gedrückt auf gedrückt) und released (Taste wechselte Zustand von gedrückt auf nicht gedrückt).

Der Zustand lässt sich über Methoden wie \textprog{isKeyDown} abrufen.

\section{InputActions}

D

\todo{InputActions}