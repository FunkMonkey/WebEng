\chapter{GameCore}

Der Kern eines Spiels befindet sich innerhalb einer vom Programmierer erweiterten (oder geerbten) Instanz der Klasse \textprog{BaseGame}. Diese Klasse beherbergt Funktionen zum Laden eines Levels sowie zum Start und Update der Game-Loop.

\section{Game-Loop}

Anders als in anderen Programmiersprachen wie C++, wo man eine Game-Loop lediglich als \textprog{while}-Schleife implementieren würde, welche das jeweils nächste Update berechnet, brechen moderne Browser Schleifen, die eine gewisse Zeitdauer überschreiten, ab. Dies hat den Grund, dass diese sonst den Main-Thread und somit die UI des Browsers (bzw. zumindest des HTMLs innerhalb der Website) blockieren würden.

Eine Möglichkeit in Javascript eine Funktion -  wie für eine Game-Loop nötig - innerhalb eines gewissen Intervals aufzurufen, bietet die DOM-Funktion \textbf{\href{http://www.w3.org/TR/html5/timers.html\#dom-windowtimers-setinterval}{setInterval}}\footnote{siehe Spezifikation unter \url{http://www.w3.org/TR/html5/timers.html\#dom-windowtimers-setinterval}}:

\begin{lstlisting}[language=JavaScript, caption=Implementierung einer Game-Loop mit \textprog{setInterval}]
function update()
{
	// calculate delta-time and do game-update
	// ...
}

// starting the game-loop: update will be called every ~16ms
window.setInterval(1/60, update);
\end{lstlisting}

Der Aufruf von \textprog{setInterval} ist aufgrund der Browser-Implementierung jedoch leider nicht sehr konstant und verzeichnet große Latenzen. \textprog{setInterval} ist daher nicht für die Implementierung einer Game-Loop geeignet.

Seit HTML5 bieten einige Browser-Hersteller jedoch einen weiteren Weg, und zwar den Aufruf einer Methode mit Hilfe von \textbf{\href{http://www.w3.org/TR/animation-timing/\#requestAnimationFrame}{requestAnimationFrame}}\footnote{siehe Spezifikation unter \url{http://www.w3.org/TR/animation-timing/\#requestAnimationFrame}}. Übergibt man \textprog{requestAnimationFrame} eine Methode, so wird diese beim nächsten Zeichnen aufgerufen und kann ihrerseits wiederum ein neues \textprog{requestAnimationFrame} aufrufen. Der Zeitpunkt des Aufrufs wird dabei vom Browser bestimmt und ist in der Regel sehr viel konstanter.

\begin{lstlisting}[language=JavaScript, caption=Implementierung einer Game-Loop mit \textprog{requestAnimationFrame}]
function update()
{
	// calculate delta-time and do game-update
	// ...
	
	// request another frame
	window.requestAnimationFrame(update);
}

// starting the game-loop: update will be called from browser-timer
window.requestAnimationFrame(update);
\end{lstlisting}